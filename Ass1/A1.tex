\documentclass[12pt,fleqn,a4paper]{article}
\usepackage[condensed]{tgheros}\renewcommand*\familydefault{\sfdefault}
%\usepackage{times}
\usepackage{mathptmx}
\usepackage{color,colordef}

%%% Set text are: \textwidth, \textheight, \topmargin, \oddsidemargin, and \evensidemargin
\textheight=10.2in\topmargin=-1.0in
\textwidth=6.5in\oddsidemargin=-0.1in\evensidemargin=\oddsidemargin

%%% Set spacings
\parskip=2mm plus 1mm minus 1mm
\parindent=0pt

%%% Customized definitions
\def\dquote{{\tt\char34}}
\def\${{\tt\char36}}
\def\bs{{\tt\char92}}
\def\^{{\tt\char94}}
\def\_{{\tt\char95}}
\def\tilde{{\tt\char126}}

\begin{document}
\begin{center}
\color{Navy}\bf
CS39003 Compilers Laboratory, Autumn 2024--2025\\
Assignment No: 1\\
Date: 05-Aug-2024

\hrulefill
\end{center}

\vspace*{-1.5\bigskipamount}
\begin{center}
\bf\color{red}Using LEX on \LaTeX\ files
\end{center}
\vspace*{-\bigskipamount}

\bigskip
\LaTeX\ is a typesetting utility written by Leslie Lamport. It is basically a macro
package designed on top of Donald Knuth's \TeX\ typesetting system. In this
assignment, you are required to write a lex program and a C/C++ program to identify some
patterns (commands, environments, math formulas, and comments) in \LaTeX\ files.
The input to your program will be a \LaTeX\ file, and the output will be the printing
of some statistics (explained near the end) associated with these patterns in the input
file.

A \TeX\ or \LaTeX\ file is a text file that stores the instructions on the text,
the mathematical formulas, the tables, the figures,
and so on, that you want to typeset. Standard text is written as it is. Special
formatting requires using commands (also called macros). \LaTeX\ additionally
provides some environments for handling a set of specialized items in the
document.

The \LaTeX\ patterns relevant to this assignment are explained in detail now.



\hrulefill

\begin{description}
\item[\LaTeX~commands]
There are three types of commands based on the characters used to define them.
\begin{itemize}
\item First, there is an active character \tilde\ which, in itself, is a command
(its default meaning is a single space where a line cannot be broken).

Other commands begin with the special character \bs (backward slash).
\item If the command name
consists only of alphabetic letters, then the \bs\ will be followed by
one of more of the letters a--z and A--Z. Some examples are \bs a, \bs abcd,
\bs aone, \bs aiii. \LaTeX\ commands are case-sensitive. For example, \bs abc,
\bs Abc, and \bs ABC are considered different commands. Notice that such
command names must contain alphabetic letters only. For example, in
\bs abc12 or \bs abc\_def, we have the command \bs abc. What follows is
not part of the command name.
\item Any character other than the alphabetic letters or \tilde\ followed
by \bs\ gives a single-character command. For example, you can have
commands like \bs\_, \bs1, \bs\bs, \bs\$, and so on, whereas in \bs\dquote u
or \bs\dquote\dquote, the command is \bs\dquote.
\end{itemize}

\item[\LaTeX~environments]
A \LaTeX\ environment begins with the special command \bs begin\{env\_name\}
and ends with \bs end\{env\_name\}. For example, for typesetting right-justified
text, one may use the flushright environment as below.

\qquad \bs begin\{flushright\}\\\mbox{}%
\qquad Write your right-justified text here.\\\mbox{}%
\qquad \bs end\{flushright\}

You can have spaces between \bs begin or \bs end, and the brace-delimited
environment name, but spaces inside the braces are not permitted.

\item[Inline math formulas]
An inline mathematical formula is written between a pair of dollars.
For example, the Van~der~Waals equation $\left(P+\frac{\alpha}{V^2}\right)
(V-\beta)=RT$ can be typeset as follows.

\qquad\$\bs left(P+\bs frac\{\bs alpha\}\{V\^2\}\bs right)(V-\bs beta)=RT\$

Note that the character \$ toggles the math mode. Inside the math mode,
there may be commands (like \bs left or \bs frac). You need to locate these
commands as well. Note also that a math-mode formula may span across
multiple lines, so you must not search for the terminating \$ in the
same line as the beginning \$.

\item[Displayed math formulas]
If you want to show your mathematical formulas in separate lines,
you can use a pair of \$\$ in the toggle mode, or the matching
commands \bs[ and \bs]. For example,

\qquad \$\$\bs left(P+\bs frac\{\bs alpha\}\{V\^2\}\bs right)(V-\bs beta)=RT\$\$

produces
% Display using $$
$$
\left(P+\frac{\alpha}{V^2}\right) (V-\beta)=RT
$$

whereas

\qquad \bs[\\\mbox{}%
\qquad ~~~~\bs left(P+\bs frac\{\bs alpha\}\{V\^2\}\bs right)(V-\bs beta)=RT\\\mbox{}%
\qquad \bs]

produces
% Display using \[ and \]
\[
\left(P+\frac{\alpha}{V^2}\right) (V-\beta)=RT
\]

One may write displayed equations across multiple lines in \LaTeX\ files,
so the begin and the end may appear in two different lines.

\item[Handling comments]
\LaTeX~does not support multi-line comments. If an (unescaped) \% appears anywhere
in a line, the rest of the line (including the new-line character at the end) is
ignored. Your lex program must ignore whatever appears in a comment. For example,
commands or math-mode toggles inside comments must be ignored. The
literal \% is printed by \bs\%, and this is not to be treated as a comment.

\item[Ignoring other text]
Your lex program must ignore texts containing none of the above patterns,
and will print nothing against these.
\end{description}

\vspace*{-\bigskipamount}
\hrulefill

\bigskip
{\bf Your LEX program}

Write appropriate regular expressions and actions for detecting the
\LaTeX\ patterns as mentioned above. You may use suitable \#define's and
global variables at the beginning of your lex program. Compile the lex code
with {\tt lex} or {\tt flex}. This will generate the C file {\tt lex.yy.c}
as usual.

\bigskip
{\bf Your C/C++ program}

{\tt\#include "lex.yy.c"} in your C/C++ program (henceforth called {\it your
C program}). You may use the compilation flag {\tt-ll} to avoid the error caused by
{\tt yywrap()}, or define the function yourself.

Inside your C program, maintain all the commands and environment names that
will be encountered during a run. Use a data structure of your own. Any data
structure to store and search strings will do. But you must not use any
STL data structure. Use one table to store the commands, and another table
to store only the environment names (without the \bs begin or \bs end, and
the braces). Against each string (command or environment name), store
a count of how many times the item is used in the \LaTeX\ source.
There is no need to keep the tables sorted (with respect to names or
the occurrence counts or first/last occurrence instants).

Also, keep two separate counts: one for storing the number of inline
mathematical formulas, and the other for storing the number of displayed
mathematical formulas.

At the end, print only the overall statistics (see the following sample output).
Without these, your program must print nothing.

\bigskip
{\bf What to submit}

Submit two files: Your lex file (like {\tt latex.l}) and your C file (like
{\tt procltx.c} or {\tt procltx.cpp}).

\newpage
{\bf Sample output}

The output on the \LaTeX\ file of this document follows.
The occurrence counts are shown within parentheses.

\begin{quote}
\color{DarkGreen}\footnotesize
\input A1sample.tex
\end{quote}

\vspace*{-\medskipamount}

\end{document}
